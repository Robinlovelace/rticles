\documentclass[11pt]{article}


\usepackage{erc}
\usepackage{multirow}

%
% Common defs for ERC proposal
%
\usepackage[square,sort,comma,numbers]{natbib} % Arguments are to fix natbib when used with numerical bib style
\usepackage{xspace}
\usepackage{doi}
\usepackage{graphicx}
\usepackage{eurosym} % for the Euro symbol
\usepackage{multirow}
\usepackage{wrapfig}
\usepackage{enumitem} % for changes to list formatting http://mirror.ox.ac.uk/sites/ctan.org/macros/latex/contrib/enumitem/enumitem.pdf
\usepackage[table]{xcolor}
\usepackage{longtable}

% Make figure captions a little smaller
\usepackage[font=small]{caption}

\usepackage{hyperref} % Should come last


\hypersetup{
   colorlinks,
   menucolor=black,
   linkcolor=black,
   citecolor=black,
   urlcolor=blue
}

\newcommand\project{DUST\xspace}
\newcommand\surname{Malleson}

\begin{document}

\lhead{\emph{\surname}}
\chead{Template Title}
\rhead{\project}
\cfoot{\thepage}

\newcommand{\note}[1]{{\color{red}\it #1}}
%\newcommand{\note}[1]{}

\begin{center}
  \vbox{\vspace{1.5cm}}
  \Large{\textbf{%
      ERC Starting Grant 2017\\
      Research proposal [Part B2]\\}
  }
  \vspace{1cm}
\end{center}

%\note{15 pages\\
%  Description of the scientific and technical aspects of the project,
%  demonstrating the ground-breaking nature of the research, its potential
%  impact and research methodology. The proposal will also need to clearly
%  specify the percentage of the applicant's total working time that will be
%  spent in the EU or an Associated Country and the percentage of the
%  applicant’s total working time that will be devoted to the project, as well
%  as a full estimation of the real project cost.
%}

% \tableofcontents

\section{State-of-the-art and objectives}

\noindent\fbox{
    \parbox{0.95\textwidth}{ This research will create a new method for dynamically assimilating data into agent-based models. This will create a step-change in our ability to reliably simulate urban systems and to forecast of the impacts of civil emergencies (and their management plans) on human populations. } }

% ************************************************************
\subsection*{Introduction}

Blahblahblah.

\hypertarget{adding-an-rmarkdown-template}{%
\subsection{Adding an RMarkdown
Template}\label{adding-an-rmarkdown-template}}

This file is what a user will see when they select your template. Make
sure that you update the fields in the yaml header. In particular you
will want to update the \texttt{output} field to whatever format your
template requires.

This is a good place to demonstrate special features that your template
provides. Ideally it should knit out-of-the-box, or at least contain
clear instructions as to what needs changing.

Finally, be sure to remove this message!

% \input{B2/method}
% \input{B2/resources}

\newpage
\appendix
%\include{ethics}


%\bibliographystyle[round]{plainnat}
%\bibliographystyle{unsrt}
\bibliographystyle{acm}
\bibliography{B1-final}

\end{document}

%%% Local Variables:
%%% mode: latex
%%% TeX-master: t
%%% TeX-PDF-mode: t
%%% End:
