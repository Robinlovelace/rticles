\documentclass[11pt]{article}


%%
%% ERC grant style file
%% Originally based on http://www.arj.no/2013/02/03/erc-stg-latex/
%% But modded a lot
%%
%% Phil Garner, Autumn 2015
%%

% Redefine sections to be the usual size, and say "Section x"
\makeatletter
\renewcommand\section{%
  \@startsection {section} % Name
  {1} %Level
  {\z@} %Indent
  {-1.5ex \@plus -1ex \@minus -.2ex} % Beforeskip
  {0.01ex \@plus.2ex} % Afterskip
  {\normalfont\large\bfseries} % Style
}
\renewcommand\subsection{%
  \@startsection{subsection}{2}{\z@}%
  {-1.25ex\@plus -1ex \@minus -.2ex}%
  {0.01ex \@plus -1ex}%
  {\normalfont\small\bfseries} % Style
}
\makeatother
\renewcommand\thesection{Section \alph{section}:}

% Header & footer
\usepackage{fancyhdr}
\pagestyle{fancy}
\setlength{\headheight}{5mm}
\setlength{\headsep}{3mm}
\setlength{\footskip}{8mm}
\renewcommand{\headrulewidth}{0pt} % Remove line at top

% Page geometry
\usepackage[a4paper,left=20mm,right=20mm,top=15mm,bottom=15mm]{geometry}
\fancyhfoffset[E,O]{0pt} % recalculate the headers

% Itemize from arj
\usepackage{enumitem}
\setitemize{noitemsep,topsep=0pt,parsep=0pt,partopsep=0pt,leftmargin=*}
\renewcommand{\labelitemi}{--}

% Misc
\usepackage{parskip}

%%% Local Variables:
%%% mode: latex
%%% End:

\usepackage{multirow}

%
% Common defs for ERC proposal
%
\usepackage[square,sort,comma,numbers]{natbib} % Arguments are to fix natbib when used with numerical bib style
\usepackage{xspace}
\usepackage{doi}
\usepackage{graphicx}
\usepackage{eurosym} % for the Euro symbol
\usepackage{multirow}
\usepackage{wrapfig}
\usepackage{enumitem} % for changes to list formatting http://mirror.ox.ac.uk/sites/ctan.org/macros/latex/contrib/enumitem/enumitem.pdf
\usepackage[table]{xcolor}
\usepackage{longtable}

% Make figure captions a little smaller
\usepackage[font=small]{caption}

\usepackage{hyperref} % Should come last


\hypersetup{
   colorlinks,
   menucolor=black,
   linkcolor=black,
   citecolor=black,
   urlcolor=blue
}

\newcommand\project{FTMOD\xspace}
\newcommand\surname{Lovelace}

\begin{document}

\lhead{\emph{\surname}}
\chead{Future transport models: data-driven, modular and open}
\rhead{\project}
\cfoot{\thepage}

\newcommand{\note}[1]{{\color{red}\it #1}}
%\newcommand{\note}[1]{}

\begin{center}
  \vbox{\vspace{1.5cm}}
  \Large{\textbf{%
      ERC Starting Grant 2018\\
      Research proposal {[}Part B2{]}\\}
  }
  \vspace{1cm}
\end{center}

%\note{15 pages\\
%  Description of the scientific and technical aspects of the project,
%  demonstrating the ground-breaking nature of the research, its potential
%  impact and research methodology. The proposal will also need to clearly
%  specify the percentage of the applicant's total working time that will be
%  spent in the EU or an Associated Country and the percentage of the
%  applicant’s total working time that will be devoted to the project, as well
%  as a full estimation of the real project cost.
%}

% \tableofcontents

\section{State-of-the-art and objectives}

\noindent\fbox{
    \parbox{0.95\textwidth}{This is the abstract.

It consists of two paragraphs.} }

% ************************************************************

\hypertarget{vision}{%
\subsection*{Vision}\label{vision}}
\addcontentsline{toc}{subsection}{Vision}

The foundations of transport models were laid in the mid
20\textsuperscript{th} century when the priority then was planning for
explosive growth in car ownership and use (Boyce and Williams 2015).
Since then much has changed: policy, research, data and technology
landscapes have been transformed. But the tools and research methods
underlying transport models have, with a few notable exceptions, been
slow to evolve.

21\textsuperscript{st} century transport planning tools must meet
different objectives. In the \textbf{European context} they must support
legally binding commitments to: tackle air pollution (see
\href{https://uk-air.defra.gov.uk/air-pollution/uk-eu-policy-context}{uk-air.defra.gov.uk});
mitigate greenhouse gas emissions (set out in the
\href{http://www.legislation.gov.uk/ukpga/2008/27/contents}{Climate
Change Act 2008}); and ensure transport equality for all (set out in the
\href{https://www.gov.uk/guidance/equality-act-2010-guidance}{Equality
Act 2010}).

The outputs of this fellowship will have \textbf{global impact} (see
Pathways to Impact). \textbf{Worldwide demand} for new transport models
is driven by such factors, the most important of which is emerging
research showing that transport policies can have huge public health
benefits, if designed correctly (Celis-Morales et al. 2017). For such
wider objectives to be met new transport modelling tools are needed:
this need is reflected in the conclusions of the Commission on Travel
Demand, the second headline recommendation of which was that tools for
transport planning should be open source (Marsden et al. 2018).

A wider driver of demand for new tools is that the transport sector is
rapidly evolving. Big Data and the uptake of Machine Learning are
parallel trends that demand new approaches, methods and software
(Lovelace et al. 2016; Lovelace and Fridman-Rojas 2017). The growth of
Data Science in industry has not only become an area of economic growth,
it has also created a skills gap following the continuing shift towards
digital technology in the transport sector, as highlighted in the
Transport Systems Catapult's
\href{http://ts.catapult.org.uk/imskills/}{Intelligent Mobility Skills
Strategy}.

Finally, transport planning is inevitably part of the democratic
process. New transport tools provide an opportunity for meaningful
engagement via two way, peer-to-peer and asynchronous communication
technologies (notwithstanding the risks of the digital divide). An
example of the potential benefits of open access and participatory
tools, and an indication of possible directions of travel, is the
Department for the Transport funded Propensity to Cycle Tool (PCT)
(Lovelace et al. 2017). The PCT meets an urgent need for public sector
and private transport planning organisations to have access to high
quality and actionable evidence with minimum barriers to entry (see
www.pct.bike). However, the approach is limited because it focusses only
on a single mode and does not allow for locally specific scenarios to be
developed.

This proposal will meet the national, global, data and citizen science
needs of transport tools outlined above. These needs mean that
21\textsuperscript{st} Century tools must take account of new datasets,
open source software packages and stakeholder needs. The rest of this
document explains how the project will do so and in the process generate
value in economic, social, environmental and health terms for decades to
come.

\hypertarget{background}{%
\section{Background}\label{background}}

\hypertarget{references}{%
\subsection*{References}\label{references}}
\addcontentsline{toc}{subsection}{References}

\hypertarget{refs}{}
\leavevmode\hypertarget{ref-boyce_forecasting_2015}{}%
Boyce, David E., and Huw C. W. L. Williams. 2015. \emph{Forecasting
Urban Travel: Past, Present and Future}. Edward Elgar Publishing.

\leavevmode\hypertarget{ref-celis-morales_association_2017}{}%
Celis-Morales, Carlos A, Donald M Lyall, Paul Welsh, Jana Anderson,
Lewis Steell, Yibing Guo, Reno Maldonado, et al. 2017. ``Association
Between Active Commuting and Incident Cardiovascular Disease, Cancer,
and Mortality: Prospective Cohort Study.'' \emph{BMJ}, April, j1456.
\url{https://doi.org/10.1136/bmj.j1456}.

\leavevmode\hypertarget{ref-lovelace_big_2016}{}%
Lovelace, Robin, Mark Birkin, Philip Cross, and Martin Clarke. 2016.
``From Big Noise to Big Data: Toward the Verification of Large Data Sets
for Understanding Regional Retail Flows.'' \emph{Geographical Analysis}
48 (1): 59--81. \url{https://doi.org/10.1111/gean.12081}.

\leavevmode\hypertarget{ref-lovelace_new_2017}{}%
Lovelace, Robin, and Ilan Fridman-Rojas. 2017. ``New Tools of the Trade?
The Potential and Pitfalls of'Machine Learning'and'DAGs' to Model
Origin-Destination Data.'' In \emph{GeoComputation 2017}, edited by Rob
Long. Centre for Computational Geography, University of Leeds.

\leavevmode\hypertarget{ref-lovelace_propensity_2017}{}%
Lovelace, Robin, Anna Goodman, Rachel Aldred, Nikolai Berkoff, Ali
Abbas, and James Woodcock. 2017. ``The Propensity to Cycle Tool: An Open
Source Online System for Sustainable Transport Planning.'' \emph{Journal
of Transport and Land Use} 10 (1).
\url{https://doi.org/10.5198/jtlu.2016.862}.

\leavevmode\hypertarget{ref-marsden_all_2018}{}%
Marsden, Greg, Elaine Seagriff, Peter Jones, Nicola Spurling, and John
Dales. 2018. \emph{All Change? The Future of Travel Demand and the
Implications for Policy and Planning. First Report of the Commission on
Travel Demand}. DEMAND.

% \input{B2/method}
% \input{B2/resources}

\newpage
\appendix
%\include{ethics}


%\bibliographystyle[round]{plainnat}
%\bibliographystyle{unsrt}
\bibliographystyle{acm}
\bibliography{B1-final}

\end{document}

%%% Local Variables:
%%% mode: latex
%%% TeX-master: t
%%% TeX-PDF-mode: t
%%% End:
